\documentclass{article}
\usepackage{amsmath,amsfonts,amsthm,amssymb}
\usepackage{setspace}
\usepackage{paralist}
\usepackage{fancyhdr}
\usepackage{lastpage}
\usepackage{extramarks}
\usepackage{chngpage}
% \usepackage{soul,color}
\usepackage{graphicx,float,wrapfig}
\usepackage{mathrsfs}
\usepackage{algorithm}
\usepackage{algorithmic}
% \usepackage{tikz}
% \usepackage{pgffor}
\newcommand{\xd}{\leqslant}
\newcommand{\dd}{\geqslant}
% In case you need to adjust margins:
\topmargin=-0.45in      %
\evensidemargin=0in     %
\oddsidemargin=0in      %
\textwidth=6.5in        %
\textheight=9.0in       %
\headsep=0.25in         %

% Setup the header and footer
\pagestyle{fancy}                                                       %
% \lhead{\StudentName}                                                 %
\chead{\Title}  %
%\rhead{\firstxmark}                                                     %
\lfoot{\lastxmark}                                                      %
\cfoot{}                                                                %
\rfoot{Page\ \thepage\ of\ \protect\pageref{LastPage}}                          %
\renewcommand\headrulewidth{0.4pt}                                      %
\renewcommand\footrulewidth{0.4pt}                                      %

\newcommand{\Proof}{\ \\\textbf{Proof:} }
\newcommand{\Answer}{\ \\\textbf{Answer:} }
\newcommand{\Acknowledgement}[1]{\ \\{\bf Acknowledgement:} #1}

\makeatletter
\newcommand{\rmnum}[1]{\romannumeral #1}
\newcommand{\Rmnum}[1]{\expandafter\@slowromancap\romannumeral #1@}
\makeatother
%%%%%%%%%%%%%%%%%%%%%%%%%%%%%%%%%%%%%%%%%%%%%%%%
%%%%%%%%%%%%%%%%%%%%%%%%%%%%%%%%%%%%%%%%%%%%%%%%
% Make title
\newcommand{\Class}{Operating System}
\newcommand{\ClassInstructor}{Xu Wei}

% Homework Specific Information. Change it to your own
\newcommand{\Title}{Nachos Phase 1 Design Document}
\newcommand{\DueDate}{March 11, 2014}
\title{\textmd{\bf \Class: \Title}\\{\large Instructed by \textit{\ClassInstructor}}\\\normalsize\vspace{0.1in}\small{Due\ on\ \DueDate}}
\date{}

\author{%
  Huang JiaChen 2011012358 \and
  Wu YueXin 2011012061 \and
  Yang Sheng 2011012359 \and
  Yin HeZheng 2011012343 \and
  Zhou XuRen 2011012353}
\newcommand{\StudentClass}{Yao Class}

% \author{\textbf{\StudentName}}
%%%%%%%%%%%%%%%%%%%%%%%%%%%%%%%%%%%%%%%%%%%%%%%%%%%%%%%%%%%%%

  \begin{document}
  \begin{spacing}{1.1}
    \maketitle \thispagestyle{empty}

%%%%%%%%%%%%%%%%%%%%%%%%%%%%%%%%%%%%%%%%%%%%%%%%%%%%%%%%%%%%%%%%%%%%%%%%%%%%%%%%%%%%%%%% Begin edit from here

\theoremstyle{plain} \newtheorem{computational}{Definition}
    \section{Task 1}

    \subsection{Correctness Constraints}
    \begin{enumerate}
      \item[$\bullet$] Caller thread waits (sleep) until callee thread finishes.
      \item[$\bullet$] If callee thread has finishes, \texttt{join()} just returns
	directly.
      \item[$\bullet$] All the threads except caller thread should not be effected.
    \end{enumerate}

    \subsection{Declarations}
    Here, we can see that in order to implement \texttt{join()} method, we need callee
    thread sends a signal to caller thread once callee finished. Therefore, we use
    semaphore to reach this goal.
    \begin{enumerate}
      \item[$\bullet$] \textit{data structures}: \texttt{Semaphore}.
      \item[$\bullet$] \textit{member variable}:
	\begin{enumerate}
	  \item \textbf{private} \texttt{Semaphore KThread::finishSemaphore}, initially 0;
	\end{enumerate}
	indicate whether the thread is finished: \texttt{value} = 0 means unfinished, while
	\texttt{value} = 1 means finished.

    \end{enumerate}

    \subsection{Description}
    because \texttt{KThread::finishSemaphore} is a private
    member variable, we need to modify its value in methord \texttt{finish()}.
    The pseudo code is as follows:
    \begin{algorithm}
      \caption{\texttt{KThread::join()}}
      \begin{algorithmic}[1]
	\STATE /* the given code */
	\STATE finishSemaphore.P();
	\STATE finishSemaphore.V();
      \end{algorithmic}
    \end{algorithm}

    \begin{algorithm}
      \caption{\texttt{KThread::finish()}}
      \begin{algorithmic}[1]
	\STATE /* the given code except sleep() */
	\STATE finishSemaphore.V();
	\STATE sleep(); // this code is the last line in the orginal KThread::finish()
      \end{algorithmic}
    \end{algorithm}

    \subsection{Testing Plan}
    \begin{enumerate}
      \item[] case 1: Only 2 thread, \texttt{thread1} and \texttt{thread2}.
	\texttt{thread1} calls \texttt{thread2.join()} before \texttt{thread2} finishes.
      \item[] case 2: Only 2 thread, \texttt{thread1} and \texttt{thread2}.
	\texttt{thread1} calls \texttt{thread2.join()} after \texttt{thread2} finishes.
      \item[] case 3: Only 3 thread, \texttt{thread1},\texttt{thread2} and
	\texttt{thread3}. \texttt{thread1} calls \texttt{thread2.join()} while
	\texttt{thread2} calls \texttt{thread3.join()}.
    \end{enumerate}

    \section{Task 2}
    \subsection{Overview}
    In this task, we have to implement condition variables directly, by using enable and disable to provide atomicity instead semaphores. To be specific, we need to implement three functions \texttt{sleep()}, \texttt{wake()} and \texttt{wakeAll()} in \texttt{Condition2} as public functions.

    As for \texttt{sleep()}, we can first disable interrupts in order to make a small critical section, in which we only need to put it in a \texttt{ThreadQueue()} and announce it to sleep.

    And for \texttt{wake()}, it can be realized by taking the first element in the queue out, and call the \texttt{ready()} function.

    \texttt{wake()} can be done by running \texttt{wake()} iteratively.
    \subsection{Correctness Constraints}
    \begin{asparaitem}
    \item When \texttt{sleep()} is called, the thread is blocked in the critical section by disabling interrupts while no thread can wake it before the \texttt{KThread.sleep()} function.
    \item When \texttt{wake()} and \texttt{wakeAll()} are performed, we guarantee all of their instructions are done within a critical section surrounded by disabling and enabling interrupts.
    \end{asparaitem}
    \subsection{Declarations}
    \texttt{waitQueue} is a \texttt{ThreadQueue} for threads blocked by \texttt{sleep()}.
    \subsection{Description}
    \begin{algorithm}
      \caption{ \texttt{void Condition2::sleep()}}
      \begin{algorithmic}
	\STATE Disable interrupts
	\STATE Release \texttt{conditionLock}
	\STATE \texttt{waitQueue.waitForAccess(KThread.currentThread())}
	\STATE	\texttt{KThread.sleep()}
	\STATE Acquire \texttt{conditionLock}
	\STATE Enable interrupts
      \end{algorithmic}
    \end{algorithm}

    \begin{algorithm}
      \caption{ \texttt{void Condition2::wake()}}
      \begin{algorithmic}
	\STATE Disable interrupts
	\IF {\texttt{thread = waitQueue.nextThread() != null}}
	\STATE \texttt{thread.ready()}
	\ENDIF
	\STATE Enable interrupts
      \end{algorithmic}
    \end{algorithm}

    \begin{algorithm}
      \caption{ \texttt{void Condition2::wakeAll()}}
      \begin{algorithmic}
	\STATE Disable interrupts
	\WHILE {\texttt{thread = waitQueue.nextThread() != null}}
	\STATE \texttt{thread.ready()}
	\STATE \texttt{thread = waitQueue.nextThread()}
	\ENDWHILE
	\STATE Enable interrupts
      \end{algorithmic}
    \end{algorithm}

    \subsection{Testing Plan}
    We plan to run with function \texttt{selfTest()}, and the specific cases are as follows:
    \begin{asparaitem}
    \item Case 1: Two threads in which one is sleeper and the other is the waker.
    \item Case 2: Several sleepers with one waker.
    \item Case 3: Several sleepers together with several wakers.
    \end{asparaitem}
    \section{Task 3}
    \subsection{Overview}
    In this task, we need to finish the public functions \texttt{waitUntil(long x)} and \texttt{timerInterrupt()} in \texttt{Alarm}. The key point is that we have to invent some efficient method instead busy waiting in \texttt{waitUntil(long x)}. Since the \texttt{timerInterrupt()} handler can be called every dozens of clock ticks, we can drive threads to sleep in \texttt{waitUntil(long x)} and decide whether they should awake in \texttt{timerInterrupt()}.
    \subsection{Correctness Constraints}
    \begin{asparaitem}
    \item The methods of \texttt{waitUntil(long x)} and \texttt{timerInterrupt()} should be atomic so that the sleeping process can not be disturbed.
    \item A queue of waiting threads should be well-maintained by handler \texttt{timerInterrupt} and \texttt{waitUntil(long x)}. To be specific, every time \texttt{waitUntil(long x)} is called, it should insert the thread into a proper place in the queue and Handler \texttt{timerInterrupt} should properly dequeue every thread out when it's invoked.
    \end{asparaitem}
    \subsection{Declarations}
    \begin{asparaitem}
    \item Class: \texttt{AlarmThread} is a wrapper that associates \texttt{KThread} with wake up time.
    \item Instance: \texttt{queue} is a priority queue \texttt{PriorityQueue<AlarmThread>} that keeps a record of sleeping threads.
    \end{asparaitem}

    \subsection{Description}
    \begin{algorithm}
      \caption{ \texttt{void Alarm::timerInterrupt()}}
      \begin{algorithmic}
	\STATE Disable interrupts
	\WHILE {\texttt{(queue.poll() != null) \&\& (queue.peek().wakeTime < Machine.timer().getTime())}}
	\STATE \texttt{queue.poll().thread.ready()}
	\ENDWHILE
	\STATE Enable interrupts
      \end{algorithmic}
    \end{algorithm}

    \begin{algorithm}
      \caption{ \texttt{void Alarm::waitUntil(long x)}}
      \begin{algorithmic}
	\STATE Disable interrupts
	\STATE \texttt{long wakeTime = Machine.timer().getTime() + x}
	\STATE Insert \texttt{new AlarmThread(Kthread.currentThread() ,wakeTime)} into \texttt{queue}
	\STATE \texttt{Kthread.sleep()}
	\STATE Enable interrupts
      \end{algorithmic}
    \end{algorithm}
    \subsection{Testing Plan}
    \begin{asparaitem}
    \item Case 1: one thread with wait time $0$.
    \item Case 2: one thread with wait time within $500$.
    \item Case 3: one thread with wait time beyond $500$.
    \item Case 4: several threads with arbitrary wait time.
    \end{asparaitem}

    \section{Task 4}

    \subsection{Correctness Constraints}
    \begin{enumerate}
      \item[$\bullet$] An active speaker should wait until an listener becomes active and
	recieves the word.
      \item[$\bullet$] At any time, there should be at most one active speaker.
      \item[$\bullet$] If there are some waiting speakers but no active speaker,
	some waiting speaker should become active.
      \item[$\bullet$] An active listener should wait until an active speaker becomes
	active and the listener recieves the speaker's word.
      \item[$\bullet$] At any time, there should be at most one active listener.
      \item[$\bullet$] If there are some waiting listeners but no active listener,
	some waiting listener should become active.
      \item[$\bullet$] Once the listener recieves the word, both the speaker and
	the listen should leave.
    \end{enumerate}

    \subsection{Declarations}
    Here we can see that the \texttt{Communicator} is a mutex resource, we need use
    a lock to protect this resource. And we can see that we need to maintain waiting
    speakers and listeners, we can use condition variables to manage them. And we
    can find that active speaker and active listener need to cooperate with other
    threads, we also use condition variables to manage them. Finially, to check whether
    there is active speaker or listener, we use two boolean variables to indicate
    respectively; and we use boolean variable to indicate whether the word is recieved.
    \begin{enumerate}
      \item[$\bullet$] \textit{data structure}: \texttt{Semaphore}, \texttt{Lock}.
      \item[$\bullet$] \textit{member variable}:
	The following variables are all the member variables of \texttt{Communicator},
	all of them are private.
	\begin{enumerate}
	  \item \texttt{Lock lock};
	  \item \texttt{Condition waitingSpeakerQueue}, initially \texttt{lock};
	  \item \texttt{Condition waitingListenerQueue}, initially \texttt{lock};
	  \item \texttt{Condition activeSpeaker}, initially \texttt{lock};
	  \item \texttt{Condition activeListener}, initially \texttt{lock};
	  \item \texttt{boolean noActiveSpeaker}, initially \texttt{true};
	  \item \texttt{boolean noActiveListener}, initially \texttt{true};
	  \item \texttt{boolean recieve}, initially \texttt{false};
	  \item \texttt{int word};
	\end{enumerate}
    \end{enumerate}

    \subsection{Description}

    \begin{algorithm}
      \caption{\texttt{Communicator::speak(w)}}
      \begin{algorithmic}[1]
	\STATE lock.acquire();
	\IF{!noActiveSpeaker}
	\STATE waitingSpeakerQueue.sleep();
	\ENDIF
	\STATE noActiveSpeaker $\leftarrow$ false;
	\STATE word $\leftarrow$ w;
	\IF{noActiveListener or !recieve}
	\STATE activeListener.wake();
	\STATE activeSpeaker.sleep();
	\ENDIF
	\STATE noActiveSpeaker, noActiveListener $\leftarrow$ true;
	\STATE recieve $\leftarrow$ false;
	\STATE waitingSpeakerQueue.wake();
	\STATE waitingListenerQueue.wake();
	\STATE lock.release();
      \end{algorithmic}
    \end{algorithm}

    \begin{algorithm}
      \caption{\texttt{Communicator::listen()}}
      \begin{algorithmic}[1]
	\STATE lock.acquire();
	\IF{!noActiveListener}
	\STATE waitingListenerQueue.sleep();
	\ENDIF
	\STATE noActiveListener $\leftarrow$ false;
	\IF{noActiveSpeaker}
	\STATE activeListener.sleep();
	\ENDIF
	\STATE activeSpeaker.wake();
	\STATE result $\leftarrow$ word;
	\STATE lock.release();
	\RETURN result;
      \end{algorithmic}
    \end{algorithm}

    \subsection{Testing Plan}
    \begin{enumerate}
      \item[] case 1: 1 speaker $\texttt{thread1}$ and 1 delay listener $\texttt{thread2}$,
	and check if $\texttt{thread1}$ wait for $\texttt{thread2}$.
      \item[] case 2: 2 speaker $\texttt{thread1}$,$\texttt{thread2}$ and
	1 listener $\texttt{thread3}$, and check if $\texttt{thread2}$ is blocked.
      \item[] case 3: exchange the position between speaker and listener in case 1 and 2.
      \item[] case 4: 2 speaker and 2 listener and check if they form two pairs.(namely,
	speaker $i$ only speaks to listen $i$, $i=1,2$)
    \end{enumerate}
\section{Task 5}
\subsection{Overview}
Task 5 asks us to implement priority scheduling in Nachos by completing the \texttt{PriorityScheduler} class. Particularly, in choosing which  thread to dequeue, the scheduler always choose a thread of the highest effective priority (choose the one that has been waiting in the queue longest if multiple threads with the same highest priority are waiting).

The most difficulty lies in the function \texttt{Scheduler.getEffectivePriority}. To calculate one thread's effective priority (abbr. EP), we have to maintain a priority queue representing all threads it is getting priority from. Notice that when a thread's EP changes, threads' EP which that thread is waiting on may change as well. So we also need to maintain which thread is it waiting on for each thread.

Specifically, we calculate a thread's prior it  by taking the max of the donor's and the recipient's priority. As for the transitive property of priority donation, we simply call \texttt{Scheduler.getEffectivePriority} recursively.

To speed up the calculation of EP, we create a variable to cache EP in class \texttt{ThreadState}. Thus, we only need to recalculate a thread's effective priority when it is possible for it to change.

\subsection{Correctness Constraints}

\begin{asparaitem}
\item All threads waiting in a queue are sorted by their effective priority\\
\item All threads store a linked list of queues they own and its own effective priority for cache\\
\item The effective priority of a thread is calculated by taking max between its initial priority and the largest effective priority stored in the linked list of queues.\\
\item When a thread is selected from the queue to acquire the resource (\texttt{nextThread}) that thread is removed from the queue and the new thread is told to aquire the queue (\texttt{acquire}) The previous owner of the queue is told it no longer owns this queue, and this queue is removed from its linked list. At the same time, this queue is inserted into the linked list of new owner. The new owner of the queue gets the highest priority in the queue added to its priority cache and its effective priority updated. \\
\item When a thread is added to the waiting queue (\texttt{waitForAccess}), it may have the highest priority of all threads in the queue. The owner of the queue then needs to update its effective priority to be greater than or equal to the new waiting thread.\\
\item When a change occurs to a thread's effective priority, it must reinsert itself into the correct position in the queue and as a result may end up affecting the effective priority of that queue's owner.
\end{asparaitem}

\subsection{Declarations}

We use \texttt{LinkedList<PriorityQueue> waitBy} and \texttt{PriorityQueue waitOn} to store threads waiting us and threads we are waiting separately. The class \texttt{PriorityQueue} is implemented by java.util.PriorityQueue.

\subsection{Descriptions}

The following algorithm describes the implementation of \texttt{getEffectivePriority} in high level.

\begin{algorithm}
  \caption{calculate effective priority}
\begin{algorithmic}[1]
  \STATE Initialize EP and donatorEP as default priority
  \STATE EP $\leftarrow$ priority
\FORALL {wait queues whose resource has been acquired by me}
  \STATE donatorEP $\leftarrow$ largest EP in that wait queue
  \IF {donatorEP $>$ EP}
    \STATE EP $\leftarrow$ donatorEP
  \ENDIF
\ENDFOR
\STATE recursively call calculate effective priority on the thread I am waiting on
\RETURN EP
\end{algorithmic}
\end{algorithm}

To clarify how we manipulate data structure, we illustrate the implementation of function \linebreak \texttt{PriorityQueue.waitForAccess}, \texttt{PriorityQueue.acquire} and \texttt{PriorityQueue.nextThread} (actually, \linebreak \texttt{waitForAccess} and \texttt{acquire} are implemented in class \texttt{ThreadState}).

\begin{algorithm}
  \caption{\texttt{ThreadState.wairForAccess(PriorityQueue waitQueue)}}
\begin{algorithmic}[1]
  \STATE waitQueue.add(this)  \COMMENT{add this thread to waitQueue}
  \STATE this.waitingOn $\leftarrow$ waitQueue  \COMMENT{update waitingOn queue}
  \STATE this.EP $\leftarrow$ calculateEffectivePriority()  \COMMENT{update EP}
\RETURN
\end{algorithmic}
\end{algorithm}

\begin{algorithm}
  \caption{\texttt{ThreadState.acquire(PriorityQueue waitQueue)}}
\begin{algorithmic}[1]
  \STATE waitQueue.owner $\leftarrow$ this
  \STATE this.waitingBy.insert(waitQueue)
  \STATE this.EP $\leftarrow$ calculateEffectivePriority()  \COMMENT{update EP}
\RETURN
\end{algorithmic}
\end{algorithm}

\begin{algorithm}
  \caption{\texttt{nextThread}}
\begin{algorithmic}[1]
  \STATE Initialize chosenThread as null
  \STATE chosenThread $\leftarrow$ pickNextThread()
  \IF {chosenThread != null}
    \STATE remove this queue from current owner's waitingBy list
    \STATE chosenThread.waitingOn $\leftarrow$ null
    \STATE insert this queue into chosenThread's waitingBy list
  \ENDIF
  \STATE this.owner $\leftarrow$ chosenThread
\RETURN chosenThread
\end{algorithmic}
\end{algorithm}

\subsection{Testing Plan}
We need to test the following things. The test is done in \texttt{selfTest()}
\begin{itemize}
\item The PriorityQueue of \texttt{java.util.PriorityQueue} works well(Automatically)
\item The transfer of priority works well
\item The transfer of priority will hold its property when queue updated
\item \texttt{Acquire()} will work well, and add queue to the queue list
\end{itemize}

\begin{algorithm}
  \caption{\texttt{selfTest}}
\begin{algorithmic}[1]
  \STATE Initialize a PriorityScheduler \texttt{s}
  \STATE Initialize 3 ThreadQueue
  \STATE Create 5 threads
  \STATE disable interrupt
  \STATE
  \STATE queue1.acquare(thread1)
  \STATE queue2.acquire(thread1)
  \STATE queue1.waitForAccess(thread2)
  \STATE queue3.acquire(thread3)
  \STATE queue3.waitForAccess(thread1)
  \STATE print all the information of the threads
  \STATE
  \STATE thread2.setPriority(3)
  \STATE print all the information of the threads
  \STATE
  \STATE queue2.waitForAccess(thread4)
  \STATE thread3.setPriority(5)
  \STATE print all the information of the threads
  \STATE
  \STATE thread3.setPriority(2)
  \STATE print all the information of the threads
  \STATE
  \STATE restore interrupt
\end{algorithmic}
\end{algorithm}

\section{Task 6}

\subsection{Design Decisions}
Let's look at the problem more closely. Consider the simplest case where there are 2 children and 1 adult on Oaku. The only feasible way to carry them to Molokai is as follows:
\begin{asparaitem}
  \item Two children go to Molokai, one rowing the boat and the other riding it.\\
  \item One child rows the boat back to Oahu.\\
  \item The adult rows to Molokai.\\
  \item The child left on Molokai rows the boat back to Oahu.\\
  \item Two children go to Molokai.\\
\end{asparaitem}

Compare this to the scheme to carry 3 children:
\begin{asparaitem}
  \item Two children go to Molokai, one rowing the boat and the other riding it.\\
  \item One child rows the boat back to Oahu.\\
  \item Two children go to Molokai.\\
\end{asparaitem}

Ignoring the two final steps of these two subroutines, those two subroutines carry one adult and one child from Oahu to Molokai respectively, using two children as a medium. Therefore, once we have two children and a boat on Oahu, we can repeatedly carry other adults and children to Molokai using the two subrountines above until there are only two children left on Oaku, then let them go across the sea by themselves.\\

Having understood this, it is reasonable to assign roles to all the adults and children. More specificly, we have the following role assignment:
\begin{asparaitem}
  \item \textbf{Rower}: The child who is responsible to row the boat. We denote it by $R$.\\
  \item \textbf{Rider}: The child who is responsible to ride the boat of the rower. We denote it by $D$.\\
  \item Other people who are going to be carried by $R$ and $D$.\\
\end{asparaitem}

\subsection{Designed Strategy}
Then the main idea can be expressed as follows:
\begin{asparaitem}
  \item We store the total number of adults and children in a global variable. Each time a person is created, the counter increases by one.\\
  \item All the children try to acquire the rower role, and only one of them get it. $R$ then decrease the population counter by 1.\\
  \item Those who failed to get the rower role try to acquire the rider role, and only one of them get it. $D$ then decrease the population counter by 1. All the adults and the children who failed then waits to be carried to Molokai.\\
  \item There is a global lock which all people except for $R$ and $D$ are waiting to acquire. If one person acquired the lock, he calls $R$ and $D$ to carry him to Molokai via one of the two subrountines, depending on whether it is a child or an adult. Then it decreases the population by 1, release the lock, and returns.\\
  \item At the end no one except $R$ and $D$ is running, and the population counter is 0. Then $R$ and $D$ perform the final rowing to Molokai and returns.
\end{asparaitem}

\subsection{Pseudocode}
The global variables we need are as follows:
\begin{asparaitem}
  \item \texttt{boolean isRowerSet}: The flag indicating if $R$ is set to a thread. All other threads should be able to read this.\\
  \item \texttt{boolean isRiderSet}: The flag indicating if $D$ is assigned to a thread. All other threads should be able to read this.\\
  \item \texttt{int population}: The population counter, initially set to 0. Whenever a person is created, this counter increases by 1; whenever a person gets across the sea, the counter decreases by 1.\\
\end{asparaitem}

Furthermore, we need the following locks and condition variables to achieve the goal:\\
\begin{asparaitem}
  \item \texttt{Lock AllMove}: During initialization of the threads, no person should be able to move, as they may incorrectly determine the situation that no one is present on Oahu as the final state if the forking of threads goes too slow. Therefore, we use the lock \texttt{AllMove} to block all threads but the main one to assure safe conditioning.\\
  \item \texttt{Lock End \& Condition CVEnd}: The lock and its corresponding condition variable which informs the main thread to quit. The $R$ thread will take charge of this lock until all people are across the sea.\\
  \item \texttt{Lock RowerMove, Condition CVRowerMove \& Lock RiderMove, Condition CVRiderMove}: These two locks are designed to synchronize the protocol between $R$, $D$ and the person being carried. Note that there can only be one person who is waiting for the lock, or it will cause problem.\\
  \item \texttt{Lock BeingCarried}: The lock held by the person being carried across the sea. Note that there can only be one thread holding it.\\
  \item \texttt{Lock PollMutex}: The lock held by the child who is trying to be $R$ or $D$.\\
  \item \texttt{Semaphore Register:} This semaphore is due to ensure that all created threads has actually incremented the population counter. Whenever a thread is created, the main thread executing \texttt{begin(int,int, BoatGrader)} performs \texttt{P()} to the semaphore, and waits for the newly created thread to wake it up after it has incremented the counter. It can also be seen that this semaphore is able to protect the counter from mutual accessing. Initially it is set to 0.\\
  \item \texttt{Semaphore AllSet:} This semaphore is used by $D$ to tell the main thread that both $R$ and $D$ has been assigned. Only with this information, the main thread can release the \texttt{AllMove} lock. Initially it is set to 0.\\
\end{asparaitem}

Then the pseudocode for the main thread is as follows:

\begin{algorithm}
  \caption{Program performed by the main thread \texttt{begin(int adults, int children, BoatGrader b)}}
\begin{algorithmic}[1]
  \STATE Initialize all global variables\\
  \STATE \texttt{AllMove.acquire()}; \COMMENT{Before forking threads, first acquire the locks to make sure to block them all}\\
\STATE \texttt{End.acquire()};\\
\FOR {$i = 1$ to $adults$}
\STATE Fork a new thread running \texttt{AdultItinerary();}\\
\STATE \texttt{Register.P();}\\
\ENDFOR
\FOR {$i = 1$ to $children$}
\STATE Fork a new thread running \texttt{ChildrenItinerary();}\\
\STATE \texttt{Register.P();}\\
\ENDFOR
\STATE \texttt{AllSet.P();}\COMMENT{Wait for $D$ to release the semaphore}\\
\STATE \texttt{AllMove.release();}\COMMENT{When woke up by already elected $D$, Unblock all threads and start crossing the river}\\
\STATE \texttt{CVEnd.sleep();}\COMMENT{Enables $R$, thus starts crossing the sea}\\
\STATE \texttt{End.release();}\COMMENT{When woke up by $R$ and informed that crossing is over, prepares to return}\\
\RETURN
\end{algorithmic}
\end{algorithm}

\begin{algorithm}
\caption{Program performed by an adult}
\begin{algorithmic}[1]
  \STATE \texttt{population ++;}\\
  \STATE \texttt{Register.V();}\\
  \STATE \texttt{AllMove.acquire();}\\
  \STATE \texttt{AllMove.release();}\COMMENT{Releases to wake up still waiting threads}\\
  \STATE \texttt{BeingCarried.acquire();}\\
  \STATE \texttt{isAdult}=\TRUE;\\
  \STATE \texttt{RowerMove.acquire();}\COMMENT{Controls $R$ to perform the protocol for adult to cross the sea}\\
  \STATE \texttt{CVRowerMove.wake();}\\
  \STATE \texttt{CVRowerMove.sleep();}\\
  \STATE \texttt{bg.AdultRowToMolokai();}\\
  \STATE \texttt{CVRowerMove.wake();}\\
  \STATE \texttt{CVRowerMove.sleep();}\\
  \STATE \texttt{population --;}\\
  \STATE \texttt{CVRowerMove.wake();}\\
  \STATE \texttt{RowerMove.release();}\\
  \STATE \texttt{BeingCarried.release();}\\
  \RETURN\\
\end{algorithmic}
\end{algorithm}


\begin{algorithm}
\caption{Program performed by children}
\begin{algorithmic}[1]
  \STATE \texttt{population ++;}\\
  \STATE \texttt{Register.V();}\\
  \STATE Set \texttt{boolean isRower} = \FALSE, \texttt{isRider} = \FALSE;\\
  \STATE \texttt{PollMutex.acquire();}\COMMENT{Trying to be $R$ or $D$}\\
  \IF {\texttt{isRowerSet}=\FALSE}
   \STATE \texttt{isRowerSet}=\TRUE;\\
  \STATE \texttt{isRower}=\TRUE;\\
  \ELSIF {\texttt{isRiderSet}=\FALSE}
  \STATE \texttt{isRiderSet}=\TRUE;\\
  \STATE \texttt{isRider}=\TRUE;\\
  \ENDIF
  \STATE \texttt{PollMutex.release();}\\
  \IF {\texttt{isRower}=\TRUE}
  \STATE \texttt{Rower();}\\
  \RETURN
  \ELSIF {\texttt{isRider}=\TRUE }
  \STATE \texttt{Rider();}
  \RETURN
  \ELSE
  \STATE \texttt{AllMove.acquire();}\\
  \STATE \texttt{AllMove.release();}\\
  \STATE \texttt{BeingCarried.acquire();}\\
  \STATE \texttt{isAdult}=\FALSE;\\
  \STATE \texttt{RowerMove.acquire();}\\
  \STATE \texttt{CVRowerMove.wake()};\\
  \STATE \texttt{CVRowerMove.sleep();}\\
  \STATE \texttt{bg.ChildrenRideToMolokai();}\\
  \STATE \texttt{CVRowerMove.wake();}
  \STATE \texttt{CVRowerMove.sleep();}\\
  \STATE \texttt{population --;}\\
  \STATE \texttt{CVRowerMove.wake();}\\
  \STATE \texttt{RowerMove.release();}
  \STATE \texttt{BeingCarried.release();}
  \RETURN
  \ENDIF
\end{algorithmic}
\end{algorithm}

\begin{algorithm}
  \caption{The method \texttt{Rower()}}
  \begin{algorithmic}

  \STATE \texttt{End.acquire();}\COMMENT{The $R$ thread is taking charge of the end condition lock}\\
  \STATE \texttt{RowerMove.acquire();}\\
  \STATE \texttt{RiderMove.acquire();}\\
  \WHILE{\texttt{population} $\geq 3$}
  \STATE \texttt{CVRowerMove.sleep();}\COMMENT{Prepares to carry people across the sea}\\
  \IF{\texttt{isAdult}=\TRUE}
  \STATE \texttt{bg.ChildRowToMolokai();}\\
  \STATE \texttt{CVRiderMove.wake();}\\
  \STATE \texttt{CVRiderMove.sleep();}\\
  \STATE \texttt{CVRowerMove.wake();}\\
  \STATE \texttt{CVRowerMove.sleep();}\\
  \STATE \texttt{bg.ChildRowToOahu();}\\
  \STATE \texttt{CVRowerMove.wake();}\\
  \STATE \texttt{CVRowerMove.sleep();}\\
  \ELSE
  \STATE \texttt{bg.ChildRowToMolokai();}\\
  \STATE \texttt{CVRowerMove.wake();}\\
  \STATE \texttt{CVRowerMove.sleep();}\\
  \STATE \texttt{bg.ChildRowToOahu();}\\
  \STATE \texttt{CVRowerMove.wake();}\\
  \STATE \texttt{CVRowerMove.sleep();}\\
  \ENDIF
  \ENDWHILE
  \STATE \texttt{RowerMove.release();}\COMMENT{Wakes up the last person}\\
  \STATE \texttt{bgChildRowToMolokai();}\\
  \STATE \texttt{CVRiderMove.wake();}\\
  \STATE \texttt{CVRiderMove.sleep();}\\
  \STATE \texttt{RiderMove.release();}\\
  \STATE \texttt{CVEnd.wake();}\\
  \STATE \texttt{End.release();}\\
  \RETURN
  \end{algorithmic}
\end{algorithm}

\begin{algorithm}
  \caption{The method \texttt{Rider()}}
  \begin{algorithmic}
    \STATE \texttt{AllSet.V();}\COMMENT{Tell the main thread that \texttt{AllMove} can be released now}\\
  \STATE \texttt{AllMove.acquire();}\\
  \STATE \texttt{AllMove.release();}\\
  \STATE \texttt{RiderMove.acquire();}\\
  \STATE \texttt{CVRiderMove.sleep();}\COMMENT{Ready to be called by $R$}\\
  \WHILE {\texttt{population} $\geq 3$}
  \STATE\texttt{bg.ChildRideToMolokai();}\\
  \STATE\texttt{bg.ChildRowToOahu()};\\
  \STATE\texttt{CVRiderMove.wake();}\\
  \STATE\texttt{CVRiderMove.sleep();}\\
  \ENDWHILE
  \STATE\texttt{bg.ChildRideToMolokai();}\\
  \STATE\texttt{CVRiderMove.wake();}
  \STATE\texttt{RiderMove.release();}
  \RETURN

  \end{algorithmic}
\end{algorithm}

\end{spacing}
\end{document}