\documentclass{article}
\usepackage{amsmath,amsfonts,amsthm,amssymb}
\usepackage{listings}
\usepackage{setspace}
\usepackage{paralist}
\usepackage{fancyhdr}
\usepackage{lastpage}
\usepackage{soul,color}
\usepackage{ulem}
\usepackage{extramarks}
\usepackage{chngpage}
\usepackage{indentfirst}
\usepackage{soul,color}
\usepackage{graphicx,float,wrapfig}
\usepackage{mathrsfs}
\usepackage[T1]{fontenc} % to be able to use < and > directly
\definecolor{pblue}{rgb}{0.13,0.13,1}
\definecolor{pgreen}{rgb}{0,0.5,0}
\definecolor{pred}{rgb}{0.9,0,0}
\definecolor{pgrey}{rgb}{0.46,0.45,0.48}
\lstset{
  showspaces=false,
  showtabs=false,
  breaklines=true,
  showstringspaces=false,
  breakatwhitespace=false,
  commentstyle=\color{pgreen},
  keywordstyle=\color{pblue},
  stringstyle=\color{pred},
  basicstyle=\ttfamily,
  moredelim=[il][\textcolor{pgrey}]{$$},
  moredelim=[is][\textcolor{pgrey}]{\%\%}{\%\%}
}
% \usepackage{algorithm}
% \usepackage{algorithmic}
% \usepackage{algpseudocode}
\usepackage[ruled, vlined, lined, boxed, commentsnumbered,linesnumbered]{algorithm2e}
\usepackage[pdfauthor={XeLaTeX},%
pdftitle={Final Report}]{hyperref}

% \usepackage{tikz}
% \usepackage{pgffor}
\renewcommand{\gets}{%
  \ensuremath{\leftarrow}}
% In case you need to adjust margins:
\topmargin=-0.45in      %
\evensidemargin=0in     %
\oddsidemargin=0in      %
\textwidth=6.5in        %
\textheight=9.0in       %
\headsep=0.25in         %

% Setup the header and footer
\pagestyle{fancy}                                                       %
% \lhead{\StudentName}                                                 %
\chead{\Title}  %
%\rhead{\firstxmark}                                                     %
\lfoot{\lastxmark}                                                      %
\cfoot{}                                                                %
\rfoot{Page\ \thepage\ of\ \protect\pageref{LastPage}}                          %
\renewcommand\headrulewidth{0.4pt}                                      %
\renewcommand\footrulewidth{0.4pt}                                      %

\newcommand{\Proof}{\ \\\textbf{Proof:} }
\newcommand{\Answer}{\ \\\textbf{Answer:} }
\newcommand{\Acknowledgement}[1]{\ \\{\bf Acknowledgement:} #1}

\makeatletter
\newcommand{\rmnum}[1]{\romannumeral #1}
\newcommand{\Rmnum}[1]{\expandafter\@slowromancap\romannumeral #1@}
\makeatother
%%%%%%%%%%%%%%%%%%%%%%%%%%%%%%%%%%%%%%%%%%%%%%%%
%%%%%%%%%%%%%%%%%%%%%%%%%%%%%%%%%%%%%%%%%%%%%%%%
% Make title
\newcommand{\Class}{Operating System}
\newcommand{\ClassInstructor}{Xu Wei}

% Homework Specific Information. Change it to your own
\newcommand{\Title}{Nachos Phase 2 Final Report}
\newcommand{\DueDate}{April 22, 2014}
\title{\textmd{\bf \Class: \Title}\\{\large Instructed by \textit{\ClassInstructor}}\\\normalsize\vspace{0.1in}\small{Due\ on\ \DueDate}}
\date{}

\author{%
  Huang JiaChen 2011012358 \and
  Wu YueXin 2011012061 \and
  Yang Sheng 2011012359 \and
  Yin HeZheng 2011012343 \and
  Zhou XuRen 2011012353}
\newcommand{\StudentClass}{Yao Class}

\definecolor{myyellow}{RGB}{255, 255, 60}

% \author{\textbf{\StudentName}}
%%%%%%%%%%%%%%%%%%%%%%%%%%%%%%%%%%%%%%%%%%%%%%%%%%%%%%%%%%%%%

  \begin{document}
  \begin{spacing}{1.1}
    \maketitle \thispagestyle{empty}

%%%%%%%%%%%%%%%%%%%%%%%%%%%%%%%%%%%%%%%%%%%%%%%%%%%%%%%%%%%%%%%%%%%%%%%%%%%%%%%%%%%%%%%% Begin edit from here

    
\section{Task 2}

\subsection{Correctness Constraints}
\begin{enumerate}
  \item[$\bullet$] Different processes do not overlap in their physical memory usage.
  \item[$\bullet$] Make use of ``gaps'' in the free memory pool.
  \item[$\bullet$] Only valid virtual memory can be read; only valid and writable
    memory can be written.
  \item[$\bullet$] All of a process's memory is freed on exit.
\end{enumerate}

\subsection{Declarations}
We use a public static \textsf{Queue} to maintain the free physical pages and a public
static \textsf{Semaphore} to make the access sychronous. Once we allocate a
page, we delete it from the queue; and once we free a page, we add it
into the queue. We use a 
\textsf{TranslationEntry} array to construct the pageTable. In the 
\textsf{UserProcess.loadSections()} we build the pageTable to associate vpn and ppn.
We implement \textsf{UserProcess.unloadSection} to release the physical pages
allocated by $\textsf{loadSections()}$.
Then we modify \textsf{UserProcess.readVirtualMemory} and 
\textsf{UserProcess.writeVirtualMemory} to use pageTable to access physical
memory through virtual memory. Here, we transfer the data in a unit of one page.
\begin{enumerate}[$\bullet$]
  \item \textit{data structure}: \textsf{Queue}, \textsf{Semaphore},
    \textsf{TranslationEntry}.
  \item \textit{member variable}:
    \begin{enumerate}
      \item \textsf{public static Queue<Integer> UserKernel::freePPNList};
      \item \textsf{public static Semaphore UserKernel::ppnListSemaphore}, initially 1;
      \item \textsf{protected TranslationEntry[] UserProcess::pageTable};
      \end{enumerate}
\end{enumerate}

\subsection{Description}
\begin{algorithm}
  \caption{\textsf{UserKernel::initialize()}}
  $\dots$ \textit{given code} $\dots$\;
  \textsf{ppnListSemaphore.P()}\;
  \For{all ppn $i$}{
    \textsf{freePPNList.enqueue(i)}\;
  }
  \textsf{ppnListSemaphore.V()}\;
\end{algorithm}

\begin{algorithm}
  \caption{\textsf{UserProcess::loadSections()}}
  \If {\textsf{numPages $>$ freePPNList.size()} }{
    \Return \textbf{false}\;
  }
  \textsf{ppnListSemaphore.P()}\;
  \textsf{pageTable.size $\gets$ numPages}\;
  \For{all section \textsf{s} and all virtual pages \textsf{vpn} in \textsf{s}}{
    \textsf{ppn $\gets$ freePPNList.dequeue()}\;
    \textsf{pageTable[vpn] $\gets$ (vpn,ppn,true,s.isReadOnly(),false,false)}\;
    Load the $\textsf{vpn}$ page's content into the physical memory page $\textsf{ppn}$\;
  }
  \For{all stack pages $\textsf{vpn}$}{
    \textsf{ppn $\gets$ freePPNList.dequeue()}\;
    \textsf{pageTable[vpn] $\gets$ (vpn,ppn,true,false,false,false)}\;
  }
  {\color{red}\textsf{vpn $\leftarrow$ vpn(argmentPage)}\;}
  {\color{red}\textsf{pageTable[vpn] $\gets$ (vpn,ppn,true,false,false,false)}\;}
  \textsf{ppnListSemaphore.V()}\;
  \Return \textbf{true}\;
\end{algorithm}

\begin{algorithm}[H]
  \caption{\textsf{UserProcess::unloadSections()}}
  \textsf{ppnListSemaphore.P()}\;
  \For{all entry \textsf{i} of \textsf{pageTable}}{
    \textsf{freePPNList.enqueue(pageTable[i].ppn)}\;
  }
  \textsf{ppnListSemaphore.V()}\;
  {\color{red}\textsf{coff.close()}\;}
\end{algorithm}

\begin{algorithm}
  \caption{\textsf{UserProcess::readVirtualMemory(vaddr, data, offset, length)}}
  $\dots$ some constrains of the parameters in the given code $\dots$ \;
  \If {\textsf{vaddr < 0 {\color{red}|| vpn(vaddr) $\ge$ pageTable.size()}}} {
    \Return 0\;
  }
  \textsf{start $\gets$  vpn(vaddr)}\;
  \textsf{end $\gets$  vpn(vaddr+length)}\;
  \textsf{en $\gets \min$\{end, pageTable.size()\}}\;
  \textsf{amount $\gets$  0};
  \For{\textsf{$\textsf{i}=$start} \textbf{to} \textsf{en}}{
    \If{\textsf{pageTable[$\textsf{i}$].valid = \textbf{false}}}{
      \Return{\textsf{amount}}\;
    }
    \textsf{s $\gets$ 0, e $\gets$ pageSize}\;
    \If{\textsf{i$=$start}}{
     \textsf{s $\gets$  offset(vaddr)}\;
    }
    \If{\textsf{i$=$end}}{
      \textsf{e $\gets$  offset(vaddr+length)}\;
    }
    \textsf{paddr = ppn*pageSize + s}\;
    \textsf{System.arraycopy(memory, paddr, data, offset, e-s)}\;
    \textsf{pageTable[i].used $\gets$}  \textbf{true}\;
    \textsf{amount  $\gets$  amount + e - s}\;
    \textsf{offset  $\gets$  offset + e - s}\;
  }
  \Return{\textsf{amount}}\;
\end{algorithm} 
The algorithm of \textsf{writeVirtualMemory} is almost the same as 
\textsf{readVirtualMemory}.
We only need to change line 10 into: \\
\begin{center}
  \textbf{if} \textsf{pageTable[i].valid = \textbf{false} || pageTable[i].readOnly = 
  \textbf{true}} \textbf{then}
\end{center}
and add a line between line 21 and line 22 as: 
\[ \textsf{pageTable[i].dirty}  \gets  \textbf{true}; \]
and change line 21 into: \\
\[ \textsf{System.arraycopy(data, offset, memory, paddr, e-s)}; \]


\subsection{Testing Plan}
\begin{enumerate}[$\bullet$]
  \item Let two process $p_1$ and $p_2$ run at the same time and see
    their ppn, then finish $p_1,p_2$ and check free ppn.
  \item At a same vpn, we first do \textsf{writeVirtualMemory},
    and do \textsf{readVirtualMemory} and then do \textsf{writeVirtualMemory},
    and do \textsf{readVirtualMemory}.
  \item Try to write in the readonly section.
  \item Try to read the memory that not in the sections.
\end{enumerate}

\begin{lstlisting}[language=Java]
public static void selfTest() {
    	UserProcess p1 = new UserProcess();
    	UserProcess p2 = new UserProcess();
    	System.out.println("the number of free pages is " + UserKernel.freePPNList.size());

    	/*
    	 * bullet 1
    	 */
    	
    	// load
    	p1.load("halt.coff", new String[]{});
    	System.out.println("the number of pages in p1 is " + p1.numPages);
    	for(int i=0; i < p1.numPages; i++)
    		System.out.println("vpn: "+ p1.pageTable[i].vpn + "\t ppn: " + p1.pageTable[i].ppn + "\t "
    				+ "ReadOnly: " + p1.pageTable[i].readOnly);
    	System.out.println("the number of free pages is " + UserKernel.freePPNList.size());
    	
    	// change p1 to p2
    	...

    	// unload
    	p1.unloadSections();
    	System.out.println("the number of free pages is " + UserKernel.freePPNList.size());
    	
    	// change p1 to p2
    	...

    	/*
    	 * bullet 2
    	 */
    	// read/write
    	p1.load("halt.coff", new String[]{});
    	System.out.println("the number of pages in p1 is " + p1.numPages);
    	for(int i=0; i < p1.numPages; i++)
    		System.out.println("vpn: "+ p1.pageTable[i].vpn + "\t\t"
    			+ "ppn: " + p1.pageTable[i].ppn + "\t\t"
    			+ "ReadOnly: " + p1.pageTable[i].readOnly + "\t\t"
    			+ "valid: " + p1.pageTable[i].valid + "\t\t"
    			+ "used: " + p1.pageTable[i].used + "\t\t"
    			+ "dirty: "+ p1.pageTable[i].dirty);
    	
    	byte[] writer = new byte[pageSize*4];
    	byte[] reader = new byte[pageSize*4];
    	for(int i=0; i < writer.length; i++)
    		writer[i] = 1;
    	for(int i=0; i < reader.length; i++)
    		reader[i] = 0;
    	int size = 1*pageSize+1234;
    	
    	int wn = p1.writeVirtualMemory(1*pageSize+1, writer, 0, size);
    	if(wn != size)
    		System.out.println("Maybe the return value of write is wrong");
    	else
    		System.out.println("wn = size");
    	int rn = p1.readVirtualMemory(1*pageSize+1, reader, 0, size);
    	if(rn != size)
    		System.out.println("Maybe the return value of read is wrong");
    	else
    		System.out.println("rn = size");
    	
    	boolean error = false;
    	for(int i=0; i < size; i++)
    		if(reader[i] != writer[i]) {
    			System.out.println("Error: read/write error at position " + i);
    			error = true;
    			break;
    		}
    	
    	if(!error)
    		System.out.println("read/write successful");
    	
    	// change writer and write/read again
    	...

    	/*
    	 * bullet 3
    	 */
    	p1.readVirtualMemory(0, reader, 0, pageSize);
    	wn = p1.writeVirtualMemory(0, writer, 0, pageSize);
    	p1.readVirtualMemory(0, reader, pageSize, pageSize);
    	
    	error = false;
    	if(wn != 0) {
    		error = true;
    		System.out.println("the return value of writing in readOnly section is not 0");
    	}
    	
    	for(int i=0; i < pageSize; i++)
    		if(reader[i] != reader[i+pageSize]) {
    			error = true;
    			System.out.println("the readOnly section is modified");
    		}
    	
    	if(!error)
    		System.out.println("pass the bullet 3");
    	
    	/*
    	 * bullet 4
    	 */
    	
    	rn = p1.readVirtualMemory(p1.numPages*pageSize + 1, reader, 0, pageSize);
    	if(rn != 0)
    		System.out.println("read the data that doesn't belong to p1");
    	else
    		System.out.println("cannot read the untouchable data");
    	
    	// do the same for write
    	...

    }

\end{lstlisting}

My test code is in the \textsf{UserProcess.selfTest}. You can check the part that I ignore.


\section{Task 3}
\subsection{Implementation}
For \textsf{exec}, we need to create a subprocess and run its \textsf{execute()} function and create standard input output file descriptors.

As \textsf{join} deals with subprocesses and can only be used between parents and children, we need to use references to record a process's parent process and a list for its children. And when the function is called, we need to check it by using \textsf{UserKernel.currentProcess()} in case of illegal joins.

And when \textsf{exit} is called, we also need to free up memory, close open files and also inform the parent process of the information so that \textsf{join} can no longer be called. We also have to notify its subprocesses the termination. Besides, since the last process calling \textsf{exit} should invoke \textsf{Kernel.kernel.terminate()}, we should keep a global list making sure that we know which one to do the closing.

Notice that we do not have to modify \textsf{Uthread.finish()} to do the same closing, since as is suggested in file \textsf{start.s}, a normal return from \textsf{main()} would call \textsf{exit} at last. Thus, we only need to care unhandled exceptions and \textsf{exit}. And the process ID is distributed at the construction function of the process.
\subsection{Pseudocodes}
Below are pseudocode for \textsf{UserProcess} \ref{alg:UserProcess}, \textsf{handleExec} \ref{alg:handleExec}, \textsf{join} \ref{alg:join}, \textsf{handleJoin} \ref{alg:handleJoin}, \textsf{cleanUp} \ref{alg:cleanUp} and \textsf{handleExit} \ref{alg:handleExit}.

And an inner class (structure) \textsf{JoinRetVal} is defined for storing the returning value (exitStatus and whether it is an normal exit).
\begin{algorithm}
  \label{alg:UserProcess}
  \caption{\textsf{UserProcess}()}
  fileDescriptorTable = new OpenFile[16]\;
  fileDescriptorTable[0] = UserKernel.console.openForReading()\;
  fileDescriptorTable[1] = UserKernel.console.openForWriting()\;
  \colorbox{myyellow}{idMutex.P()}\;
  processID = nextProcessID\;
  nextProcessID++\;
  \colorbox{myyellow}{idMutex.V()}\;
\end{algorithm}

\begin{algorithm}[H]
  \label{alg:handleExec}
  \caption{int \textsf{handleExec}(int file, int argc, int argv)}
  fileName\gets\textsf{readVirtualMemoryString}(file,256)\;
  \If {(fileName is invalid) or (argc $<$ 0)} {
    \Return -1\;
  }
  \If {fileName does not end with \textsf{.coff}}{
    \Return -1\;
  }
  byte[] pointer = new byte[4]\;
  String[] args = new String[argc]\;
  \For {i = 0; i $<$ argc; i++}{
  numRet = \textsf{readVirtualMemory}(argv, pointer)\;
  {\color{red}\tcc{check whether args whose argv.length < args}
  \If {numRet < 4 || pointer == null} {
	\Return -1\;
  }}
    int pos = Lib.bytesToInt(pointer, 0)\;
    args[i] = \textsf{readVirtualMemoryString}(pos, args[i])\;
    \colorbox{myyellow}{// in a 4-byte alignment situation}\;
    argv += 4
  }
  UserProcess childProcess = new UserProcess()\;
  \If {childProcess is not valid} {
    \Return -1\;
  }
  childProcess.setParent(this)\;
  childProcessList.add(childProcess)\;
  \colorbox{myyellow}{tableMutex.P()}\;
  userProcessTable.put(childProcess.getID(), childProcess)\;
  \colorbox{myyellow}{tableMutex.V()}\;
  \If {not childProcess.\textsf{execute}(fileName, args)}{
    \Return -1\;
  }
  \Return childProcess.getID()\;
\end{algorithm}

\begin{algorithm}
  \label{alg:join}
  \caption{JoinRetVal \textsf{join}()}
  \tcc{note that we have to keep both exitStatus and normalExit since the exit status is not required to be consistent with a normal exit}
  joinSemaphore.P()\;
  \tcc{assume an integer is of 4 bytes and the processor is little endian}
  {\color{red}\sout{byte[] data = Lib.bytesFromInt(exitStatus)}\;
  \sout{writeVirtualMemory(status, data)}\;}
  joinSemaphore.V()\;
  {\color{red}\Return new JoinRetVal(normalExit,exitStatus)\;}
\end{algorithm}

\begin{algorithm}[H]
  \label{alg:handleJoin}
  \caption{int \textsf{handleJoin}(int processID, int status)}
  \tcc{check validity}
  \colorbox{myyellow}{tableMutex.P()}\;
  UserProcess joinProcess = userProcessTable.get(processID)\;
  \colorbox{myyellow}{tableMutex.V()}\;
  {\color{red}
  \tcc{parentProcess does not exist or not equal}
  \If {joinProcess == null || joinProcess.parentProcess == null} {
  \Return -1\;
  \ElseIf {joinProcess.parentProcess.getID() != UserKernel.currentProcess().getID()} {
  \Return -1\;
  }
  }
  JoinRetValue joinVal = joinProcess.join()\;
  int retVal = joinVal.normalExit ? 1 : 0\;
  byte[] data = Lib.bytesFromInt(joinVal.exitStatus)\;
  \tcc{write memory in parent thread}
  writeVirtualMemory(status, data)\; }
  \tcc{only need to remove the child in join since only join accesses children}
  childProcessList.remove(joinProcess)\;
  \Return retVal\;
\end{algorithm}

\begin{algorithm}
\label{alg:cleanUp}
\caption{void \textsf{cleanUp}()}
    \For {file in fileDesriptorTable}{
      \If {file != null} {
        file.close()\;
      }
    }
    \colorbox{myyellow}{//need free up memory!}\;
    unloadSections()\;
    \For {child in childProcessList}{
      \If {child != null} {
        child.setParent(null)\;
      }
    }
    \colorbox{myyellow}{tableMutex.P()}\;
    userProcessTable.remove(UserKernel.getCurrentProcess().getID())\;
    \colorbox{myyellow}{tableMutex.V()}\;
    childProcessList.clear()\;
\end{algorithm}

\begin{algorithm}
  \label{alg:handleExit}[H]
  \caption{int \textsf{handleExit}(int status, boolean normalExit)}
  \tcc{we also use this function to handle unhandled exceptions except that we manually set status}
  \textsf{cleanUp}()\;
  {\color{red}\sout{If userProcessTable.isEmpty()} \;}
  \If {processID == 0} {
    Kernel.kernel.terminate()\;
  }
  exitStatus = status\;
  this.normalExit = normalExit\;
  joinSemaphore.V()\;
  \colorbox{myyellow}{KThread.finish()}\;
  \Return status\;
\end{algorithm}
\subsection{Testing}
\begin{asparaitem}
  \item {\color{red}\sout{A program with $argc=0$.}}
  \item {\color{red}\sout{Processes killed midway to test whether the parent can receive the correct information and make sure that the cleanups are correctly done.}} This is because it is hard to interrupt threads in the middle of      executing in our model and \textsf{selfTest}().
  \item {\color{red} \sout{Console for inputs and outputs can not established.}} This is not the case we need to consider since the robustness of the console is guaranteed by the project itself.

  \item {\color{red}\sout{Interrupt at userProcessTable putting in new pairs}}
%  \item That whether use equals of java object
  \item Joining on a finished process
\end{asparaitem}

\subsection{Tests Taken}
\subsubsection{\textsf{exec} with zero argument}
This test program calls \textsf{exec} without passing any argument, i.e. \textsf{argc}=0 and \textsf{argv}=0. This program should not call \textsf{exec} successfully.
\begin{lstlisting}[language=C]
// This is a test of exec call with zero argc.
#include "syscall.h"

int main()
{
    char* proc = "halt.coff";
    int argc = 0;
    char** argv = 0;
    int pid = exec(proc, argc, argv);
    if(pid == -1)
    {
        printf("zeroargc test failed: exec executed normally without argc\n");
    }
    else
    {
        printf("zeroargc test passed\n");
    }
}
\end{lstlisting}
This test is passed correctly.
\subsubsection{testing \textsf{join}}
This test is to test whether multiple child processes can be joined by their parent successfully. Correct run of the processes outputs nothing and exits with \textsf{exitStatus}=0.
\begin{lstlisting}[language=C]
// joinDetect.c
#include "stdio.h"
#define MAXBUF 256
int main(int argc, char** argv)
{
    int exitStatus = 0, childCount = 1, i;
    char* proc = "exit";
    char agv[2][MAXBUF];
    strcpy(agv[0], proc);
    strcat(proc, ".coff");
    strcpy(agv[1], argv[1]);
    int exitStatNormal = atoi(argv[1]);
    if(argc > 2) childCount = atoi(argv[2]);
    for(i = 0; i < childCount; i ++)
    {
        int pid = exec(proc, 2, agv);
        int exitStat;
        join(pid, &exitStat);
        if(exitStatNormal != exitStat)
        {
            printf("%d\n", exitStat);
            exitStatus = -1;
        }
    }
    exit(exitStatus);
}

//exit.c
#include "stdio.h"
int main(int argc, char** argv)
{
    if(argc == 1) exit(0);
    else exit(atoi(argv[1]));
}
\end{lstlisting}

\subsubsection{parent halts when child sleeps}
This is a test where the child process keeps running and the parent halts right after forking the child. It should run and exit normally.
\begin{lstlisting}[language=C]
// sleepingParent.c
#include "stdio.h"
int main()
{
    char* proc = "sleepingChild";
    char* argv[1];
    strcpy(argv[0], proc);
    strcat(proc,".coff");
    exec(proc, 1, argv);
    exit(0);
}

// sleepingChild.c
// The busy cycles are used to simulate sleeping condition; note that we have not implemented scheduling constraints yet
#define CYCLES 3000000
int main()
{
    int i = 0;
    while(i < CYCLES) i ++;
    exit(0);
}
\end{lstlisting}


\end{spacing}
\end{document}
