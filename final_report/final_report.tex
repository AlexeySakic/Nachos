\documentclass{article}
\usepackage{amsmath,amsfonts,amsthm,amssymb}
\usepackage{listings}
\usepackage{setspace}
\usepackage{paralist}
\usepackage{fancyhdr}
\usepackage{lastpage}
\usepackage{extramarks}
\usepackage{chngpage}
\usepackage{indentfirst}
\usepackage{soul,color}
\usepackage{graphicx,float,wrapfig}
\usepackage{mathrsfs}
\usepackage[T1]{fontenc} % to be able to use < and > directly
\definecolor{pblue}{rgb}{0.13,0.13,1}
\definecolor{pgreen}{rgb}{0,0.5,0}
\definecolor{pred}{rgb}{0.9,0,0}
\definecolor{pgrey}{rgb}{0.46,0.45,0.48}
\lstset{
  showspaces=false,
  showtabs=false,
  breaklines=true,
  showstringspaces=false,
  breakatwhitespace=false,
  commentstyle=\color{pgreen},
  keywordstyle=\color{pblue},
  stringstyle=\color{pred},
  basicstyle=\ttfamily,
  moredelim=[il][\textcolor{pgrey}]{$$},
  moredelim=[is][\textcolor{pgrey}]{\%\%}{\%\%}
}
% \usepackage{algorithm}
% \usepackage{algorithmic}
% \usepackage{algpseudocode}
\usepackage[ruled, vlined, lined, boxed, commentsnumbered,linesnumbered]{algorithm2e}
\usepackage[pdfauthor={XeLaTeX},%
pdftitle={Final Report}]{hyperref}

% \usepackage{tikz}
% \usepackage{pgffor}
\renewcommand{\gets}{%
  \ensuremath{\leftarrow}}
% In case you need to adjust margins:
\topmargin=-0.45in      %
\evensidemargin=0in     %
\oddsidemargin=0in      %
\textwidth=6.5in        %
\textheight=9.0in       %
\headsep=0.25in         %

% Setup the header and footer
\pagestyle{fancy}                                                       %
% \lhead{\StudentName}                                                 %
\chead{\Title}  %
%\rhead{\firstxmark}                                                     %
\lfoot{\lastxmark}                                                      %
\cfoot{}                                                                %
\rfoot{Page\ \thepage\ of\ \protect\pageref{LastPage}}                          %
\renewcommand\headrulewidth{0.4pt}                                      %
\renewcommand\footrulewidth{0.4pt}                                      %

\newcommand{\Proof}{\ \\\textbf{Proof:} }
\newcommand{\Answer}{\ \\\textbf{Answer:} }
\newcommand{\Acknowledgement}[1]{\ \\{\bf Acknowledgement:} #1}

\makeatletter
\newcommand{\rmnum}[1]{\romannumeral #1}
\newcommand{\Rmnum}[1]{\expandafter\@slowromancap\romannumeral #1@}
\makeatother
%%%%%%%%%%%%%%%%%%%%%%%%%%%%%%%%%%%%%%%%%%%%%%%%
%%%%%%%%%%%%%%%%%%%%%%%%%%%%%%%%%%%%%%%%%%%%%%%%
% Make title
\newcommand{\Class}{Operating System}
\newcommand{\ClassInstructor}{Xu Wei}

% Homework Specific Information. Change it to your own
\newcommand{\Title}{Nachos Phase 2 Final Report}
\newcommand{\DueDate}{April 22, 2014}
\title{\textmd{\bf \Class: \Title}\\{\large Instructed by \textit{\ClassInstructor}}\\\normalsize\vspace{0.1in}\small{Due\ on\ \DueDate}}
\date{}

\author{%
  Huang JiaChen 2011012358 \and
  Wu YueXin 2011012061 \and
  Yang Sheng 2011012359 \and
  Yin HeZheng 2011012343 \and
  Zhou XuRen 2011012353}
\newcommand{\StudentClass}{Yao Class}

\definecolor{myyellow}{RGB}{255, 255, 60}

% \author{\textbf{\StudentName}}
%%%%%%%%%%%%%%%%%%%%%%%%%%%%%%%%%%%%%%%%%%%%%%%%%%%%%%%%%%%%%

  \begin{document}
  \begin{spacing}{1.1}
    \maketitle \thispagestyle{empty}

%%%%%%%%%%%%%%%%%%%%%%%%%%%%%%%%%%%%%%%%%%%%%%%%%%%%%%%%%%%%%%%%%%%%%%%%%%%%%%%%%%%%%%%% Begin edit from here


\section{Task 3}
\subsection{Tests Taken}
\subsubsection{\textsf{exec} with zero argument}
This test program calls \textsf{exec} without passing any argument, i.e. \textsf{argc}=0 and \textsf{argv}=0. This program should not call \textsf{exec} successfully.
\begin{lstlisting}[language=C]
// This is a test of exec call with zero argc.
#include "syscall.h"

int main()
{
    char* proc = "halt.coff";
    int argc = 0;
    char** argv = 0;
    int pid = exec(proc, argc, argv);
    if(pid == -1)
    {
        printf("zeroargc test failed: exec executed normally without argc\n");
    }
    else
    {
        printf("zeroargc test passed\n");
    }
}
\end{lstlisting}
This test is passed correctly.
\subsubsection{testing \textsf{join}}
This test is to test whether multiple child processes can be joined by their parent successfully. Correct run of the processes outputs nothing and exits with \textsf{exitStatus}=0.
\begin{lstlisting}[language=C]
// joinDetect.c
#include "stdio.h"
#define MAXBUF 256
int main(int argc, char** argv)
{
    int exitStatus = 0, childCount = 1, i;
    char* proc = "exit";
    char agv[2][MAXBUF];
    strcpy(agv[0], proc);
    strcat(proc, ".coff");
    strcpy(agv[1], argv[1]);
    int exitStatNormal = atoi(argv[1]);
    if(argc > 2) childCount = atoi(argv[2]);
    for(i = 0; i < childCount; i ++)
    {
        int pid = exec(proc, 2, agv);
        int exitStat;
        join(pid, &exitStat);
        if(exitStatNormal != exitStat)
        {
            printf("%d\n", exitStat);
            exitStatus = -1;
        }
    }
    exit(exitStatus);
}

//exit.c
#include "stdio.h"
int main(int argc, char** argv)
{
    if(argc == 1) exit(0);
    else exit(atoi(argv[1]));
}
\end{lstlisting}

\subsubsection{parent halts when child sleeps}
This is a test where the child process keeps running and the parent halts right after forking the child. It should run and exit normally.
\begin{lstlisting}[language=C]
// sleepingParent.c
#include "stdio.h"
int main()
{
    char* proc = "sleepingChild";
    char* argv[1];
    strcpy(argv[0], proc);
    strcat(proc,".coff");
    exec(proc, 1, argv);
    exit(0);
}

// sleepingChild.c
// The busy cycles are used to simulate sleeping condition; note that we have not implemented scheduling constraints yet
#define CYCLES 3000000
int main()
{
    int i = 0;
    while(i < CYCLES) i ++;
    exit(0);
}
\end{lstlisting}


\end{spacing}
\end{document}
